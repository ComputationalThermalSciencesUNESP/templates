\documentclass[a4paper,12pt]{article}

    \usepackage[%
        hidelinks,
        pdfauthor   ={Nome do autor},
        pdftitle    ={Título do trabalho},
        pdfproducer ={LaTeX with hyperref}
    ]   {hyperref}

    \usepackage{graphicx}

    %% Recomendação para citação padrão ABNT
    %\usepackage[%
    %    style=abnt,
    %    %backend=biber,
    %    natbib=true,
    %    isbn=false,
    %    url=false%
    %]{biblatex}

        %\addbibresource{books-refs.bib}

    \usepackage{fancyhdr}
    \usepackage{setspace}
    \usepackage{geometry}
    \usepackage{enumitem}

    \fancypagestyle{feb-header}{%
        \renewcommand{\headrulewidth}{0pt}
        %
        \fancyhf{}\fancyhead[L]{%
            \begin{minipage}{0.50\textwidth}
                \begin{flushleft}
                    \includegraphics[%
                        width=0.52\paperwidth
                    ]   {img/Logo_UNESP_horizontal_Bauru_transp.png}
                \end{flushleft}
            \end{minipage}%
            \begin{minipage}{0.50\textwidth}
                \begin{flushright}
                    \includegraphics[%
                        width=0.12\paperwidth
                    ]   {img/Logo_FEB_efeito_vertical_transp.png}
                \end{flushright}
            \end{minipage}
        }
    }

    % Change geometry
    \newgeometry{%
        headheight=1.5cm,
        headsep=0.75cm,
        top=3.0cm,
        left=1.5cm,
        right=1.5cm,
        bottom=1.5cm
    }%

    \setlength{\parindent}{0pt}

    % TROQUE OS NOMES DOS HEADERS AQUI
    \fancypagestyle{regular-header}{%
        \fancyhead[L]{%
            \textit{Anote o Título da Prática ou Trabalho Aqui}
        }
        \fancyhead[R]{%
            \textit{Nome da Disciplina}
        }
    }

    % Suporte para duas colunas
    \usepackage{multicol}

    \usepackage{sectsty}
        \sectionfont{\normalsize\bfseries}

    % Para usar Fonte Arial, descomente estas linhas
    %\usepackage{helvet}
    %\renewcommand{\familydefault}{\sfdefault}

    % Para usar Times, descomente estas linhas
    \usepackage{newtxtext}
    \usepackage{newtxmath}

    % Compile with XELatex
    %\usepackage{fontspec}
    %    \setmainfont{Arial}
    %    %\setmainfont{Times New Roman}

\begin{document}

    %\singlespacing
    \thispagestyle{feb-header}

    \begin{center}
        \large
        \textbf{%
            Relatório X -- Nome da Disciplina%
        }
    \end{center}

    Título da Prática ou Trabalho: \rule{0.707\textwidth}{0.75pt}\\
    Nome: \rule{0.73\textwidth}{0.75pt} RA: \rule{0.16\textwidth}{0.75pt}
    % Adicione mais linhas de nome se for trabalho em grupo

    \singlespacing
    \pagestyle{regular-header}

    Formatação obrigatória: fonte Arial ou Times New Roman, tamanho 12, com
    espaçamento simples entre linhas.

    % Para ativar duas colunas, adicione o texto entre o env. multicols
    \begin{multicols}{2}
        Recomendações: O formato do corpo do trabalho é livre para o discente
        mudar como achar mais importante, de acordo com as figuras, tabelas e
        equações que forem necessários. Recomendo usar duas colunas para
        aproveitar o espaço em trabalhos mais extensos.

        Com relação à seções do trabalho, elas também são livres com exceção de
        relatórios de trabalhos experimentais, nos quais as seções devem ser:

        \section{Objetivos}

            Descrever de forma sucinta os objetivos do trabalho ou
            experimento. Em um trabalho científico, os objetivos resumem a
            pergunta que estamos interessados em responder. Por exemplo: “Qual
            tipo de convecção, natural ou forçada, causa um maior coeficiente
            de transf. de calor por convecção?”

        \section{Fundamentação Teórica}

            Explicar as bases teóricas (físicas e matemáticas) que foram
            utilizadas.
        
        \section{Metodologia}

            A seção de metodologia em um trabalho científico deve conter como
            planejamos responder a pergunta feita nos objetivos. A descrição
            deve ser sucinta a ponto de que um outro grupo possa reproduzir o
            trabalho. Se o trabalho for experimental, incluir figuras das
            montagens e lista dos materiais usados. Se o trabalho for numérico
            ou computacional, descrever os algoritmos usados, por exemplo.

        \section{Resultados e Discussão}

            
            Apresentação dos resultados obtidos a partir da metodologia
            montada. Normalmente, em um trabalho científico, são gerados muito
            mais resultados do que devemos apresentar. Então, o que mostrar no
            relatório? A orientação deve ser a pergunta a ser respondida,
            proposta nos objetivos. Assim, inclua apenas os resultados que você
            usou para responder a pergunta. Tipicamente aqui são apresentadas
            as implicações dos resultados e as limitações dos resultados
            obtidos também.   

        \section{Conclusões}

            Tipicamente, a conclusão é uma seção mais enxuta que resume o que
            foi obetido neste trabalho (qual a resposta da pergunta proposta?).

        \section{Referências Bibliográficas}

            Dica: para inclusão de referências bibliográficas nos trabalhos,
            recomendo o \href{zotero.org}{Zotero}, uma ferramenta de
            gerenciamento de bibliografias.  Após instalar o Zotero no seu
            computador, você pode exportar sua lista de referências como
            arquivos .bib.
    \end{multicols}

\end{document}
